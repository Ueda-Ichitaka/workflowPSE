
\begin{document} 
	\chapter{Zielbestimmung}
	
	    \section{Einleitung}
		    
		    Die Motivation des Projektes WPSflow, welches in diesem Dokument beschrieben wird, beruht auf der \textbf{V}irtuellen \textbf{For}schungsumgebung für die \textbf{Wa}sser- und \textbf{Ter}restrische Umweltforschung (V-For-Wa-Ter, bzw. \url{http://vforwater.de/}) welche eine generische, virtuelle Forschungsumgebung für den gemeinsamen, systemischen Umgang mit Daten aus der Wasser- und Umweltforschung im Rahmen des \href{http://www.wassernetzwerk-bw.de/}{Netzwerks Wasserforschung Baden-Württemberg} ist.\\
		    Mit WPSflow wird hierfür ein Webportal entwickelt, welches den Nutzern erlaubt \Gls{Workflow}s zu erstellen, zu bearbeiten oder zu löschen. Sie können diesen Workflows auch Parameter übergeben, sie ausführen und den Ausführungsstatus in einer eigenen Übersichtsanzeige überwachen. Danach können Ergebnisse als Diagramm dargestellt werden.\\
		    Aufgrund der hohen angestrebten Nutzerfreundlichkeit wird das Portal ansprechend für eine breite Masse sein und gerne genutzt werden, weshalb wir, das \textbf{WPSflow-Team} in der Entwicklung auch viel Wert auf Stabilität und eine intuitive Bedienbarkeit legen.
		    
		    \vspace{.5cm}
		    
	    
	    %\noindent Die Anwendung WPSflow soll es Benutzern ermöglichen, \Gls{Workflow}s in einem graphischen Editor zu erstellen, bearbeiten und in den Ausführungsstatus in einer eigenen Übersichtsanzeige zu überwachen. Die Anwendung kann über das Forschungsportal \url{http://vforwater.de/} aufgerufen werden.
	    
% 		Die Benutzer des Forschungsportals \url{http://vforwater.de/} sollen durch das Produkt in die Lage versetzt werden, \Gls{Workflow}s in einem graphischen Editor zu erstellen, bearbeiten und in einer eigenen Übersichtsanzeige den Ausführungsstatus zu überwachen.
		
		    
		\section{Musskriterien}
			\begin{itemize}
				\item Verwalten von \gls{Web Processing Service} Workflows
					\begin{itemize}
						\item Erstellen, Bearbeiten, Speichern, Laden, Anzeigen von Workflows
						\item Auflistung von Workflows mit Ausführungsstatus
						\item Wiederherstellung der letzten Sitzung
						\item Fehlermanagement
						    \begin{itemize}
                            	\item Editorprüfung auf Kompatibilität von Tasks innerhalb eines Workflows
                            \end{itemize}
						\item Workflows und \Gls{Task}s auf Syntax und Kompatibilität prüfen
						\item Erstellen und Bearbeiten in graphischem \Gls{Drag and Drop} Editor
						\item Dynamisches einbinden von neuen \gls{Web Processing Service} Tasks in den Editor
					\end{itemize}
				\item Login im Portal
				\item Nutzerfreundlichkeit
					\begin{itemize}
						\item Einfache, intuitive Benutzung des Editors
					\end{itemize}
				\item Open-Source
				\item Der Nutzer hat die Möglichkeit WPS konforme Dateien als Workflows hoch zu laden
				% \item Dateiupload
				%     \begin{itemize}
				%         \item Upload von Dateien als Input für Workflow Tasks
				%     \end{itemize}
				\item Integrierte Ausführung der Workflows auf dem gleichen Server nur nach erfolgter Prüfung von bestimmten Kriterien, z.B. Serverlast
				\item Öffentliche und Benutzergebundenen Workflows
				    \begin{itemize}
				        \item Workflows mit anderen Benutzern Teilen und öffentlich zugänglich machen
				        \item Zugriff und Sichtbarkeit von Workflows auf eigenen oder einzelne Benutzer beschränken
				    \end{itemize}
				% \item Dokumentation % ich glaube die Elnaz meinte hier das war eher an unser Projekt gerichtet, nicht als Musskriterium für das Pflichtenheft
				\item Technische Dokumentation (Benutzeranleitung)
				\item Hilfesektion für Benutzerfragen
			\end{itemize}
		
		\section{Wunschkriterien}
			\begin{itemize}
				\item Logging aller Aktivitäten
				\item Import und Export von Workflows
				\item Detailansicht einzelner Workflows
				\item Erweiterte Konfigurationseinstellungen
				\item Ein extra Interface für Administratoren%Administrationsinterface
				\item Installations-/Einrichtungsassistent
			\end{itemize}	
		
		\section{Abgrenzungskriterien}
			\begin{itemize}
				\item Einsatz des Portals nicht für große Nutzermenge geplant
				\item Keine verteilte Datenbank, keine Echtzeitanforderungen
			\end{itemize}	


\end{document}
