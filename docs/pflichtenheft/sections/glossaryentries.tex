
%##!!##!!##!!##
% Liste bisheriger gls:
% [MVC], [Django], [JS], [Drag'n'Drop], [WPS]
%

%####################
% Zielbestimmung
%####################

\newglossaryentry{Workflow}
{
    name=Workflow,
    description={Ein strukturierter Arbeitsablauf.}
}

\newglossaryentry{Task}
{
    name=Task,
    description={Ein Schritt in einem Workflow.}
}

%####################
% Produkteinsatz
%####################



%####################
% Produktumgebung
%####################

\newglossaryentry{Model View Controller}
{
    name=Model View Controller,
    description={Model View Controller ist ein Architekturmuster, das in der Programmierung für einen flexiblen Entwurf eingesetzt wird um eine Anwendung einfacher an verschiedene Schnittstellen anzupassen. Nur View und Controller müssten dafür dann neu implementiert werden.}
}

\newglossaryentry{Django}
{
    name=Django,
    description={Django ist ein auf Python basierendes Open-Source-Webframework das dem \Gls{Model View Controller} Schema folgt.}
}

\newglossaryentry{JavaScript}
{
    name=JavaScript,
    description={JavaScript ist eine Skriptsprache die für dynamisches HTML in Webbrowsern entwickelt wurde.}
}

\newglossaryentry{Server}
{
    name=Server,
    description={Der Server stellt die Anwendung für den Client bereit.}
}

\newglossaryentry{Client}
{
    name=Client,
    description={Der Client stellt das Endgerät des Nutzers dar, der auf die Anwendung zugreift.}
}
\newglossaryentry{Steinbuch Centre for Computing}
{
    name=Steinbuch Centre for Computing,
    description={Das wissenschaftliche Rechenzentrum des Karlsruher Institut für Technologie.}
}

\newglossaryentry{Orgware}
{
    name=Orgware,
    description={Teile der Anwendung die weder zu Software noch zu Hardware gehören, meist organisatorischer Natur.}
}

%####################
% Funktionale Anforderungen
%####################

\newglossaryentry{Drag and Drop}
{
    name=Drag and Drop,
    description={Methode zur Bedienung grafischer Benutzeroberflächen von Rechnern durch das Bewegen grafischer Elemente mittels eines Zeigegerätes.}
}

\newglossaryentry{Web Processing Service}
{
    name=WPS,
    description={Der Web Processing Service (WPS) ist ein Mechanismus, um über das Internet eine räumliche Analyse von Geodaten durchzuführen.}
}

%####################
% Produktdaten
%####################



%####################
% Nichtfunktionale Anforderungen
%####################



%####################
% Globale Testfälle
%####################



%####################
% Systemmodelle
%####################


