
\chapter{Produktumgebung}

In diesem Kapitel wird auf die Umgebung, speziell Software, Hardware und Orgware, eingegangen. Für die Punkte Software und Hardware wird hierfür nochmal zwischen Serverseite und Clientseite unterschieden, bei \Gls{Orgware} werden nur allgemeine Punkte erwähnt. Damit soll möglichst genau erklärt werden, was benötigt wird um einen zuverlässigen Anwendungsverlauf zu erreichen.

\begin{itemize}
    \item Entwickelt wird eine Web-Anwendung, die somit nicht durch ein Betriebssystem eingeschränkt ist und mit den meisten browserfähigen Geräten kompatibel ist
    \item Die Anwendung wird später über ein schon existierendes Online-Portal erreichbar sein
    \item Die Anwendung hat eine Client-Server Architektur, die nach dem \Gls{Model View Controller}-Prinzip entworfen wird
    \item Sie wird in einem \Gls{Django} Projekt entwickelt
    \item Die Anwendung läuft auf einem \Gls{Server} des KIT
    \item Sie steht nach Entwicklungsabschluss weltweit zur Verfügung
\end{itemize}

\section{Software}

\begin{itemize}
    \item \textbf{Serverseite}
        \begin{itemize}
            \item Python3.6
            \item Django
            \item Eine Datenbank
        \end{itemize}
    \item \textbf{Clientseite}
        \begin{itemize}
            \item Der \Gls{Client} benötigt einen \Gls{JavaScript}-fähigen Browser um auf die Anwendung zuzugreifen
            \item Der Browser des Client muss JavaScript erlauben
        \end{itemize}
\end{itemize}

\newpage

\section{Hardware}

\begin{itemize}
    \item \textbf{Serverseite}
        \begin{itemize}
            \item Die Hardware wird vom \Gls{Steinbuch Centre for Computing} bereit gestellt womit die Auslastung gut verteilt ist
            \item Aufgrund der geringen erwarteten Nutzungskapazität ist eine Überlastung fast ausgeschlossen
            \item Eine Internetverbindung ist erforderlich
        \end{itemize}
    \item \textbf{Clientseite}
        \begin{itemize}
            \item Der Client benötigt ein Browser- und JavaScript-fähiges Gerät
            \item Um einen Workflow zu erstellen ist ein Zeigegerät (z.B. Maus) und eine Tastatur notwendig
            \item Eine Internetverbindung ist erforderlich
        \end{itemize}
\end{itemize}

\section{Orgware}

\begin{itemize}
    \item Für das erstellen der Workflows ist eine Internetverbindung notwendig
    \item Der Login zur Anwendung erfolgt über die bereits existierende Plattform entweder über die Plattform selbst oder über den WaTTS-Login
    \item Ein Zugriffspunkt zu der Anwendung ist das schon existierende Forschungsportal \url{http://vforwater.de/}
\end{itemize}

