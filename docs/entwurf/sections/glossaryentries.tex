%!!!!!!!!!!!!!!!!!!!!!
%   Liste bisheriger Glossary entries:
%
%   UML, Cron
%
%!!!!!!!!!!!!!!!!!!!!

\newglossaryentry{UML}
{ 
    name = UML, 
    description={Unified Modeling Language}
}
\newglossaryentry{Cron}
{
    name = Cron,
    description={Cron ist ein Programm, das ständig im Hintergrund abläuft und der zeitbasierten Ausführung von Prozessen unter unixartigen Systemen, um wiederkehrende Aufgaben zu automatisieren}
}
\newglossaryentry{REST}
{
    name = REST,
    description={Ist ein Programmierparadigma für HTTP Server. Hierbei wird eine Standard für die Kommunikation von Ressourcen festgelegt.}
}
\newglossaryentry{Workflow}
{
    name=Workflow,
    description={Ein strukturierter Arbeitsablauf}
}

\newglossaryentry{Task}
{
    name=Task,
    description={Ein Schritt in einem Workflow}
}
\newglossaryentry{Web Processing Service}
{
    name=WPS,
    description={Der Web Processing Service (WPS) ist ein Protokoll, welches über das Internet eine räumliche Analyse von Geodaten durchführt}
}
\newglossaryentry{Model View Controller}
{
    name=Model View Controller,
    description={Model View Controller ist ein Architekturmuster, das in der Programmierung für einen flexiblen Entwurf eingesetzt wird um eine Anwendung einfacher an verschiedene Schnittstellen anzupassen. Nur View und Controller müssten dafür dann neu implementiert werden}
}

\newglossaryentry{Django}
{
    name=Django,
    description={Django ist ein auf Python basierendes Open-Source-Webframework das dem \Gls{Model View Controller} Schema folgt}
}

\newglossaryentry{JavaScript}
{
    name=JavaScript,
    description={JavaScript ist eine Skriptsprache die für dynamisches HTML in Webbrowsern entwickelt wurde}
}

\newglossaryentry{Server}
{
    name=Server,
    description={Der Server stellt die Anwendung für den Client bereit}
}

\newglossaryentry{Client}
{
    name=Client,
    description={Der Client stellt das Endgerät des Nutzers und die dort ausgeführten Anwendungen dar}
}
\newglossaryentry{Steinbuch Centre for Computing}
{
    name=Steinbuch Centre for Computing,
    description={Das wissenschaftliche Rechenzentrum des Karlsruher Institut für Technologie}
}
\newglossaryentry{Drag and Drop}
{
    name=Drag and Drop,
    description={Methode zur Bedienung grafischer Benutzeroberflächen von Rechnern durch das Bewegen grafischer Elemente mittels eines Zeigegerätes}
}
